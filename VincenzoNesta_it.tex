%% start of file `template.tex'.
%% Copyright 2006-2010 Xavier Danaux (xdanaux@gmail.com).
%
% This work may be distributed and/or modified under the
% conditions of the LaTeX Project Public License version 1.3c,
% available at http://www.latex-project.org/lppl/.

% Version: 20110122-4


\documentclass[11pt,a4paper]{moderncv}

\usepackage{VincenzoNesta}

\usepackage[T1]{fontenc}
\usepackage[italian]{babel}

%fontawesome definition
%\usepackage{fontawesome}
%\newfontfamily{\FA}{FontAwesome Regular}
%\def\twitter{{\FA \faTwitter}}

%lettere accentate
\catcode`\ì=\active \def ì{\`{\i}}
\catcode`\í=\active \def í{\’{\i}}
\catcode`\è=\active \def è{\`e}
\catcode`\é=\active \def é{\’e}
\catcode`\È=\active \def È{\`E}
\catcode`\É=\active \def É{\’E}
\catcode`\à=\active \def à{\`a}
\catcode`\á=\active \def á{\’a}
\catcode`\À=\active \def À{\`A}
\catcode`\Á=\active \def Á{\’A}
\catcode`\ù=\active \def ù{\`u}
\catcode`\ú=\active \def ú{\’u}
\catcode`\ò=\active \def ò{\`o}
\catcode`\ó=\active \def ó{\’o}

\linespread{0.9}

% for some reason, lines take up a lot of space in itemize in English...
\newenvironment{tightitemize}
   {\begin{itemize}
   \setlength{\parskip}{0pt}}
   {\end{itemize}}


% personal data

\title{Curriculum Vit\ae}

\extrainfo{%
  nato a Brindisi il 26/11/1973}
%\linkedin~\httplink{www.linkedin.com/in/vinnes}\\%
%\octocat~\httplink{www.github.com/vinnes}\\}

%citazione
%\myquote{...l'aggetto delle cornici sbalzerà di la dal piombo le gocce, e con questo riguardo si verranno a mantenere salde le muraglie di mattoni}{M. Vitruvio Pollione - \itshape De Architectura}
%\nopagenumbers{}                             % uncomment to suppress automatic page numbering for CVs longer than one page
%----------------------------------------------------------------------------------
%            content
%----------------------------------------------------------------------------------
\begin{document}
%\setmainfont{Minion Pro}
%\setsansfont{Myriad Pro}

\hyphenpenalty=10000

\maketitle

% \section{Capacità e competenze}
% 
% \subsection{Capacit\`a professionali}
% \cvline{Direzione di Cantiere}{Gestione globale della Commessa, con organizzazione del cantiere, gestione manodopera e mezzi, programmazione dei lavori, conoscenza dei materiali da costruzione e delle principali tecniche costruttive oltre che di restauro conservativo e consolidamento. \linebreak
%   Inoltre preventivazione acquisti, conoscenza delle normative sia dei lavori privati che pubblici, corretti rapporti con la Committenza, DD.LL. e Responsabili del Procedimento, gestione efficace di subappaltatori e fornitori.}
% \cvline{Contabilit\`a di Cantiere}{Uso degli strumenti di misura, tenuta di brogliacci delle misure, rapidit\`a nelle operazioni di rilevo in cantiere, uso avanzato di fogli di calcolo e di software di contabilit\`a, tenuta dei registri contabili. \linebreak 
%   Inoltre studio e redazione dei Capitolati Speciali d'Appalto, individuazione dei lavori in Variante, iscrizione di eventuali riserve, analisi dei prezzi, formazione di liste per lavori in economia, studio gare di appalto.}
% \cvline{Controllo di Gestione}{Formazione del budget di commessa e del quadro economico di riferimento, controllo costi ed aggiornamento costante del quadro economico, individuazione dei ratei professionisti/fornitori/subappaltatori, individuazione S.I.L., previsone emissione S.A.L., redazione del piano finanziario e del cash-flow, individuazione Margine Operativo Lordo, reportistica avanzata sull'andamento del cantiere.}
% \cvline{Certificazione Energetica}{Audit energetico degli edifici esistenti, con acquisizione dati catastali ed esecuzione di accurato sopralluogo e rilievo. Analisi mirata dell'involucro edilizio e degli impianti. Inserimento dati e calcolo degli indici EP\ped{i}, EP\ped{acs} ed Ep\ped{GL} mediante software certificato C.T.I. Emissione, consegna e deposito dell'A.P.E. \linebreak
%   Valutazione, studio ed individuazione dei possibili interventi di riqualificazione energetica, dei risparmi economici ottenibili e dei tempi di ritorno degli investimenti. \linebreak
%   Valutazioni su progettazione/integrazione impiantistica con impianti da F.E.R. e redazione pratiche per accesso agli incentivi fiscali. Studio di soluzioni per il conseguimento di risparmi energetici, analisi e calcolo dei ponti termici, studio degli ombreggiamenti e ricorso ad analisi in regime dinamico, anche parametriche, per i casi pi\`u complessi.}
% 
% \subsection{Competenze organizzative e gestionali}
% \cvline{}{In qualità di Direttore Tecnico di Cantiere ho seguito la realizzazione di una decina di importanti appalti pubblici in Puglia. Ho gestito pi\`u cantieri in contemporanea coordinando le maestranze, gestendo con decisione i fornitori ed i subappaltatori, rapportandomi con le DD.LL., e collaborando efficacemente con gli impiegati della sede pugliese del Consorzio Etruria. \linebreak
%   Le mie decisioni erano prese in piena autonomia organizzativa, avendo conquistato sul campo la stima e la considerazione dei Direttori di Produzione con cui ho avuto l'occasione di lavorare. \linebreak
%   Nei grandi appalti privati di cui sono stato il Responsabile della Contabilità di Cantiere, ho avuto collaboratori giovani che ho sempre spronato a dare il massimo ed ho contribuito a far crescere professionalmente. \linebreak
%   L'organizzazione del mio ufficio si è sempre distinta per avere sempre aggiornato il controllo dei costi, il rispetto di tempi ed obiettivi, ed ha contribuito agli ottimi risultati economici ottenuti con la realizzazione dei tre Centri Commerciali per Unicoop Firenze.}
% 
% \subsection{Competenze comunicative}
% \cvline{}{Le mie capacità di comunicare efficacemente sia con il mio gruppo di lavoro sia con le "controparti" (Enti pubblici o committenti privati, Direzione Lavori, ditte subappaltrici o fornitrici) per me sono fondamentali per una corretta gestione tecnica ed economica degli appalti. \linebreak
%   E' da considerare inoltre che preferisco sempre mettere in chiaro, sia verbalmente sia mettendolo nero su bianco, problematiche, aspetti tecnico/economici e "regole". Lo faccio mediante riunioni periodiche con il gruppo, verifiche in contraddittorio delle lavorazioni svolte con i responsabili delle ditte. \linebreak
%   La mia propensione allo scrivere tutto quanto di rilevante per la gestione ordinata del rapporto di lavoro mi ha sempre consentito una gestione degli aspetti tecnici e della contabilità precisa ed accurata che, anche negli appalti o subappalti più complicati, hanno prodotto risultati sempre soddisfacenti.}
% 
% \subsection{Competenze informatiche}
% \cvcomputer{Sistemi Operativi}{Microsoft Windows, Apple OSX/iOS, GNU/Linux}{Office}{Word, Excel, iWork, Open/LibreOffice, GIMP}
% \cvcomputer{Disegno e Progettazione}{AutoCAD, DraftSight}{Modellazione 3D}{ArchiCAD, Trimble Sketchup, 3D Studio Max}
% \cvcomputer{Project Management}{Microsoft Project, Planner}{DataBase}{Microsoft Access, MySQL}
% \cvcomputer{Contabilità dei lavori}{ACCA Primus, STR}{Hardware}{Gestione di reti di computer, assemblaggio e configurazione PC}
% \cvcomputer{Sviluppo software}{Bash scripting language, Python, Perl, AppleScript}{Certificazione energetica}{CNR Docet, CNR Docet\ap{Pro}, Aermec MC11300, Namirial Termo}
% \cvcomputer{Ponti termici, Energy Modeling, Retrofit}{LBL Therm, EERE EnergyPlus, NREL Openstudio}{Energie rinnovabili}{RETScreen, fogli di calcolo MyGreenBuildings}

%\devnotes{Developer}{Contributor}

\section{Esperienze professionali}

% Center labels and use "Since"
%\tltextstart[base]{\scriptsize}
%\tltextend[base]{\scriptsize}
\tlsince{dal }

%oppure cambio i putni di ancoraggio
%questo mi serve per spostare in basso la data iniziale per evitare che finisca sulla barra
\renewcommand{\tltextstart}[2][south west]{%
   \tikzset{
       tl@startyear/.style={
           font=#2,
           name=tl@startyear,
           below=1pt, % it was null
           inner xsep=0pt,
           anchor=#1,
       }
   }
}
%set the anchor as you prefer
\tltextstart[north]{\scriptsize}
\tltextend[south]{\scriptsize} 
%--------


\tlcventry{2013}{0}{Certificatore Energetico}{\href{http://www.firenzenergia.it/}{Agenzia Fiorentina per l'Energia}}{Firenze}{}{Durante lo stage previsto dal corso CERT.E. di certificatore energetico organizzato da A.S.E.V.:
\begin{tightitemize}
 \item Collaborazione al progetto europeo \href{http://www.serpente-project.eu}{SERPENTE} (Surpassing Energy Targets through Efficient Public
Buildings);
 \item Redazione delle certificazioni energetiche di alcuni edifici pubblici della Provincia di Firenze;
 \item Diagnosi energetiche con l'individuazione dei possibili interventi di riqualificazione energetica degli edifici;
 \item Analisi e studio degli interventi con individuazione dei tempi di ritorno degli investimenti;
 \item Supporto all'Audit energetico nell'esecuzione dei controlli di qualità per alcuni edifici in fase di certificazione CasaClima;
 \item Certificazione energetica di edifici privati.
\end{tightitemize}}

\tlcventry{2011}{0}{Direttore Tecnico}{\href{http://www.cla1921.it/}{Cooperativa L'Avvenire 1921}}{Montelupo F.no (FI)}{}{Attualmente in Cassa Integrazione Guadagni Staordinaria.
\begin{tightitemize}
 \item Redazione delle certificazioni energetiche di alcuni edifici privati;
 \item Studio e report dei possibili interventi di riqualificazione energetica.
\end{tightitemize}}

\tlcventry{2005}{2011}{Responsabile della Contabilit\`a}{\href{http://www.consorzioetruria.it/}{Consorzio Etruria}}{Montelupo F.no (FI)}{}{Contabilit\`a attiva e passiva di importanti Centri Commerciali realizzati in Toscana per UNICOOP Firenze:%
\begin{tightitemize}
 \item Centro comm.le CENTRO*Empoli - dal 2005 al 2007;
 \item Centro comm.le PARCO*Prato - dal 2007 al 2009;
 \item Magazzino logistico Centro Freschi di Pontedera - dal 2009 al 2011;
 \item Contabilit\`a attiva verso la Committente per un totale di oltre \textbf{155} milioni di euro;
 \item Contabilit\`a passiva verso fornitori e subappaltatori per un totale di circa \textbf{128} milioni di euro;
 \item Studio e redazione di Varianti in corso d'opera per circa \textbf{25} milioni di euro;
 \item Addetto al Controllo di Gestione, con procedure per la rilevazione di S.I.L. e cost-control sviluppate personalmente;
 \item Assistente di Cantiere, con funzioni di ripreventivazione del budget di Commessa, preventivazione acquisti, controllo manodopera, Preposto alla Sicurezza.
\end{tightitemize}}

\tlcventry{2002}{2005}{Direttore Tecnico}{\href{http://www.consorzioetruria.it}{Consorzio Etruria - Direzione Fuori Sede Puglia}}{Lecce}{}{Gestione globale delle Commesse:
\begin{tightitemize}
 \item Lavori di completamento delle sistemazioni esterne dell'I.P.A. di S.ta Cesarea Terme, Stazione Appaltante Provincia di Lecce;
 \item Lavori per la costruzione di un Centro Gestionele e Posto di Ristoro nella Z.I. di Brindisi, Stazione Appaltante S.I.S.R.I. Brindisi;
 \item Lavori di costruzione della Sede decentrata degli Uffici Comunali di Gallipoli, tramite Procura speciale dell'impresa Polistrade - Campi Bisenzio (FI);
 \item Lavori di costruzione dell'I.P. Alberghiero di S.ta Cesarea Terme, tramite Procura speciale del Consorzio Cooperativo Costruzioni - Bologna;
 \item Addetto al Controllo di Gestione per tutti i cantieri afferenti alla Direzione Fuori Sede del Consorzio Etruria di Lecce.
\end{tightitemize}}

\tlcventry{1999}{2001}{Assistente di cantiere}{E.R.\& B. Consulting}{Lecce}{}{Tecnico di cantiere, gestione maestranze e mezzi, contabilit\`a di cantiere, preventivazione acquisti nei seguenti importanti lavori di restauro, cat. OG2, nel centro storico di Lecce, Stazione Appaltante Comune di Lecce:
\begin{tightitemize}%
 \item Ex Conservatorio S.Anna, Lecce;
 \item Ex Monastero dei Teatini, Lecce;
 \item Recupero di Palazzo Turrisi-Palumbo, ditta appaltatrice Faesulae;
 \item Lavori di restauro dell'ex Monastero degli Olivetani, Stazione Appaltante Universit\`a degli Studi di Lecce;
 \item Addetto all'Uffico Gare con lo studio di circa venti gare d' appalto tra il 2000 e il 2001;
 \item Progettazione, preventivazione e successiva gestione del parco hardware e software delle rete di computer degli uffici. 
\end{tightitemize}}

 \tlcventry{1998}{1999}{Disegnatore CAD}{STIGEA Cooperativa a r.l.}{L'Aquila}{}{Prestazioni occasionali nella socie\`a operante nel settore dei lavori stradali per l'ampliamento della s.s. n. 7 Brindisi-Taranto - III lotto:
\begin{tightitemize}%
 \item Rilievi topografici, restituzione su CAD, inserimento e sviluppo delle misure per la redazione della contabilit\`a e dei relativi allegati; 
\end{tightitemize}}

% Restore normal labels
%\tltext{\scriptsize}

%\pagebreak

\section{Esperienze formative}

\tldatecventry{2013}{Corso Base per Progettisti CasaClima}{\href{http://www.firenzenergia.it/afe/formazione.php}{Agenzia Fiorentina per l'Energia}}{Firenze}{}{Introduzione a CasaClima, Fondamenti di Fisica Tecnica applicata e isolamento termico, Materiali e Costruzioni, Impiantistica.}

\tlcventry{2012}{2013}{CERT.E. Corso per Certificatore Energetico}{\href{http://www.asev.it/formazione/archivio/certe.php}{ASEV}}{Empoli (FI)}{}{Corso della durata complessiva di 327 ore di cui 140 di stage. Normativa, Fondamenti di Fisica Tecnica, Materiali edili, costruzioni e isolamenti, Impianti, Metodologie di Misurazione-calcolo, Project work.}

\tlcventry{2012}{2013}{Corso in Bioedilizia e Bioarchitettura}{CESVIP Toscana-Lazio}{Firenze}{}{Bioedilizia, strategie passive per il risparmio energetico in edilizia, certificazione energetica degli edifici.}

\tldatecventry{2013}{Corso Aggiornamento di formazione dei lavoratori nel settore costruzioni - Rischio alto}{ClA 1921 s.c.}{Montelupo F.no (FI)}{}{}

\tldatecventry{2013}{Corso formativo in CAD 2D - Avanzato}{CESVIP Toscana-Lazio}{Firenze}{}{}

\tldatecventry{2012}{Corso di Tecniche di conduzione e controllo di cantieri edili}{CESVIP Toscana-Lazio}{Firenze}{}{Tecniche di conduzione e controllo di cantieri edili, allestimento del cantiere, macchine e attrezzature edili, normativa sugli appalti pubblici.}

\tldatecventry{2011}{Corso su Sistemi geotermici a bassa entalpia per applicazioni a pompa di calore}{Klimahouse Umbria 2011}{}{Bastia Umbra (PG)}{}{}

\tldatecventry{2009}{Corso per Addetti del Servizio di Prevenzione e Protezione}{Comitato Tecnico Paritetico della Toscana}{Montelupo F.no (FI)}{}{Codice ATECO 3 - costruzioni.}

\tldatecventry{2009}{Corso di formazione per la prevenzione incendi}{Comitato Tecnico Paritetico della Toscana}{Montelupo F.no (FI)}{}{}

\tldatecventry{2009}{Corso di formazione teorico-pratico per addetti al primo soccorso}{Comitato Tecnico Paritetico della Toscana}{Montelupo F.no (FI)}{}{}

\tlcventry{1992}{1996}{Studente in Ingegneria Edile}{\href{http://www.poliba.it/}{Politecnico}}{Bari}{}{Frequenza di quindici corsi del Corso di Laurea, di cui undici superati gli esami con votazione media 28/30.}

\tlcventry{1987}{1992}{Maturit\`a Tecnica per Geometri}{Istituto Tecnico Statale per Geometri ``O. Belluzzi''}{Brindisi}{}{}

\section{Certificazioni}

\tldatecventry{2013}{Tecnico della progettazione ed elaborazione di sistemi di risparmio energetico}{\href{http://www.asev.it/formazione/archivio/certe.php}{ASEV}}{Empoli (FI)}{}{ADA \#1: Attivit\`a di monitoraggio delle strutture esistenti.}

\tldatecventry{1992}{Diploma di Maturit\`a Tecnica per Geometri}{Istituto Tecnico Statale per Geometri ``O. Belluzzi''}{Brindisi}{}{Votazione 56/60.}


\section{Lingue straniere}
\cvlanguage{Italiano}{Nativo}{Madrelingua}
\cvlanguage{Inglese}{Buon livello}{Lettura abituale di testi tecnici in inglese, pratica occasionale}
\cvlanguage{Francese}{Livello intermedio}{Studiato 3 anni a scuola, pratica occasionale}
%\cvlanguage{Russian}{Intermediary Level}{Studied 3 years in school}
%\cvlanguage{Dutch}{Beginner}{Studied alone}
%\cvlanguage{Swedish}{Beginner}{Studied alone}

\section{Interessi personali}

\cvhobby{Sport}{Pallacanestro, praticata anche a livello agonistico. Istruttore di mini-basket.}
\cvhobby{Altro}{Viaggi in camper, lettura, tecnologia, cucina.}

%\renewcommand{\listitemsymbol}{-} % change the symbol for lists


% Publications from a BibTeX file without multibib\renewcommand*{\bibliographyitemlabel}{\@biblabel{\arabic{enumiv}}}% for BibTeX numerical labels
%\nocite{*}
%\bibliographystyle{plain}
%\bibliography{publications}       % 'publications' is the name of a BibTeX file

% Publications from a BibTeX file using the multibib package
%\section{Publications}
%\nocitebook{book1,book2}
%\bibliographystylebook{plain}
%\bibliographybook{publications}   % 'publications' is the name of a BibTeX file
%\nocitemisc{misc1,misc2,misc3}
%\bibliographystylemisc{plain}
%\bibliographymisc{publications}   % 'publications' is the name of a BibTeX file

%liberatoria privacy
\tiny\vspace{\fill}
\footnotesize\noindent\centering
Autorizzo il trattamento dei miei dati personali ai sensi del D.Lgs. 196/2003.
\vspace*{0.25cm}

Empoli, \today\hfill  In fede:\raisebox{-0.75cm}{\includegraphics[scale=1]{firma}}

\end{document}

