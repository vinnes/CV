%!TEX encoding = UTF-8
%!TEX program = xelatex
%!TEX spellcheck = it_IT
%% start of file `VincenzoNesta.tex'.
%% Copyright 2006-2010 Xavier Danaux (xdanaux@gmail.com).
%% Forked in 2014 from github repository https://github.com/raphink/CV  - Raphael Pinson (raphink@gmail.com).
%% Modified in 2017 by Vincenzo Nesta (vincenzo.nesta@gmail.com).
%
% This work may be distributed and/or modified under the
% conditions of the LaTeX Project Public License version 1.3c,
% available at http://www.latex-project.org/lppl/.

% Version: 20170120-1

%-------------------------------------------------------------------------------
% CONFIGURATIONS
%-------------------------------------------------------------------------------

%tipo documento
\documentclass[11pt,a4paper,nolmodern]{moderncv}

%stili, macro e opzioni personalizzate da VincenzoNesta.sty
\usepackage{VincenzoNesta}

%lingua
\usepackage[italian]{babel}

%extra
\usepackage{xltxtra}

%interlinea
\linespread{0.9}

% for some reason, lines take up a lot of space in itemize in English...
\newenvironment{tightitemize}
   {\begin{itemize}
   \setlength{\parskip}{0pt}}
   {\end{itemize}}

% personal data

\title{Curriculum Vit\ae}

\extrainfo{%
%\telegram~\httplink{t.me/vinnes73}\\
social:~~~\href{https://www.linkedin.com/in/vinnes}{\linkedin}~~~%
\href{https://www.github.com/vinnes}{\github}~~~%
\href{https://www.twitter.com/vinnes}{\twitter}~~\textbf{vinnes}\\
nato a Brindisi il 26/11/1973\\
automunito, patente B}%


%citazione
%\myquote{\ldots l'aggetto delle cornici sbalzerà di la dal piombo le gocce, e con questo\\
%riguardo si verranno a mantenere salde le muraglie di mattoni.}{\textsc{M. Vitruvio Pollione - De Architectura}}

%\nopagenumbers{}                             % uncomment to suppress automatic page numbering for CVs longer than one page

%----------------------------------------------------------------------------------
%            content
%----------------------------------------------------------------------------------

\begin{document}

%%Set Adobe fonts 
\setmainfont[Mapping=tex-text, BoldFont={Minion Pro Bold}]{Minion Pro}
\setsansfont[Mapping=tex-text, BoldFont={Myriad Pro Bold}]{Myriad Pro}

\hyphenpenalty=10000

\maketitle

%rimuovere i numeri di pagina
\thispagestyle{empty}
\pagestyle{empty}

\section{Esperienze professionali}

% Center labels and use "Since"
%\tltextstart[base]{\scriptsize}
%\tltextend[base]{\scriptsize}
\tlsince{dal }

%oppure cambio i punti di ancoraggio
%questo mi serve per spostare in basso la data iniziale per evitare che finisca sulla barra
\renewcommand{\tltextstart}[2][south west]{%
   \tikzset{
       tl@startyear/.style={
           font=#2,
           name=tl@startyear,
           below=1pt, % it was null
           inner xsep=0pt,
           anchor=#1,
       }
   }
}
%set the anchor as you prefer
\tltextstart[north]{\scriptsize}
\tltextend[south]{\scriptsize} 

%--------
%% Esperienze professionali
%--------
% COOPSINTEC
\tldatecventry{2020}{Libero professionista, addetto al controllo di gestione}{\href{https://www.sintecsocietacooperativa.com/}{SINTEC soc. coop.}}{Fornaci di Barga (LU)}{}{da settembre a novembre 2020 collaborazione a tempo pieno con le società cooperative SINTEC e Pro.Ge.Co.
\begin{tightitemize}
\item Ristudio delle commesse acquisite, principalmente lavori pubblici cat.~OG1, interventi di restauro cat.~OG2, riqualificazione energetica edifici scolastici;
\item Redazione dei budget di riferimento per le nuove commesse, ed implementazione del sistema di controllo gestione anche per i cantieri già avviati; 
\end{tightitemize}}
% BONAGUIDI&GIUSTI
\tlcventry{2019}{2020}{Libero professionista, tecnico del risparmio energetico}{}{}{Empoli}{da luglio 2019 a febbraio 2020 collaborazione a tempo pieno con la E.S.Co. (Energy Service Company) \href{http://www.bonaguidigiusti.it/}{BONAGUIDI\&GIUSTI Associati s.r.l.s.} di Empoli.
\begin{tightitemize}
 \item Elaborazione, redazione e presentazione della documentazione tecnica per il collaudo di impianti fotovoltaici per la produzione di energia elettrica e di impianti termodinamici a pompa di calore per la produzione di acqua calda sanitaria, committente principale \href{http://www.nwg.it/}{NWG s.p.a.} ;
 \item Delegato alla registrazione degli interventi di installazione, manutenzione, riparazione e smantellamento degli impianti termodinamici sul portale \href{http://www.fgas.it/}{F-gas}, ai sensi del D.P.R. 146/2018;
 \item Consulenza e raccolta della documentazione delle ditte installatrici relativa al D.Lgs. 81/2008 (Testo Unico sulla Salute e Sicurezza sul Lavoro)
\end{tightitemize}}
% Free Energy
% \tlcventry{2019}{0}{Procacciatore di affari}{\href{https://www.free-energysrl.it/}{FREE ENERGY s.r.l.}}{Arezzo}{}{Procacciamento di affari per la vendita di Energia Elettrica, Gas e Servizi di Efficientamento Energetico.}
%\begin{tightitemize}
% ESAERG (stage)
\tldatecventry{2019}{Tecnico del risparmio energetico}{\href{http://www.esaerg.it/}{ESAERG s.r.l.}}{Arezzo}{}{Durante lo stage previsto dal corso Mech-Energy organizzato da Assoservizi:
\begin{tightitemize}
 \item Redazione delle diagnosi energetiche di alcuni edifici pubblici della Provincia di Arezzo;
 \item Preparazione della documentazione tecnica da inviare tramite il PortalTermico del GSE per l'accesso agli incentivi previsti dal Conto Termico 2.0;
 \item Sviluppo di un'applicazione web-based (\href{https://github.com/vinnes/senerg}{sENERG}) per la ricerca delle imprese energivore che hanno l'obbligo di eseguire una diagnosi energetica (art. 8 D.lgs. 102/14);
\end{tightitemize}}
% mobilità volontaria, sviluppatore e tecnico risparmio energetico
\tlcventry{2016}{0}{Tecnico del risparmio energetico}{}{}{}{da giugno 2016 a dicembre 2017 in mobilità volontaria ai sensi della L. 223/91.
\begin{tightitemize}
 \item Sviluppo di un'applicazione web-based (\href{https://github.com/vinnes/openstudio_nodejs}{Retrofit Manager}) per semplificare il retrofit energetico di edifici civili, commerciali ed industriali;
 \item Diagnosi energetiche con l'individuazione degli interventi di riqualificazione energetica degli edifici;
 \item Analisi e studio degli interventi con individuazione dei tempi di ritorno degli investimenti;
 \item Certificazione energetica di edifici privati.
\end{tightitemize}}
% A.F.E. (ora A.R.R.R.) (stage)
\tldatecventry{2013}{Certificatore energetico}{\href{http://www.firenzenergia.it/}{Agenzia Fiorentina per l'Energia}}{Firenze}{}{Durante lo stage previsto dal corso CERT.E. organizzato da A.S.E.V.:
\begin{tightitemize}
 \item Collaborazione al progetto europeo \href{http://www.serpente-project.eu}{SERPENTE} (Surpassing Energy Targets through Efficient Public
Buildings);
 \item Redazione delle certificazioni energetiche di alcuni edifici pubblici della Provincia di Firenze;
 \item Analisi e studio degli interventi con individuazione dei tempi di ritorno degli investimenti;
 \item Supporto all'audit energetico nell'esecuzione dei controlli di qualità per alcuni edifici in fase di certificazione CasaClima;
 \end{tightitemize}}
% Coop. L'Avvenire
\tlcventry{2012}{2016}{Direttore tecnico}{\href{http://www.cla1921.it/}{Cooperativa L'Avvenire 1921}}{Montelupo F.no (FI)}{}{In cassa integrazione guadagni straordinaria per crisi e ristrutturazione aziendale.
\begin{tightitemize}
 \item Redazione delle certificazioni energetiche di alcuni edifici privati;
 \item Studio e report dei possibili interventi di riqualificazione energetica.
\end{tightitemize}}
% Consorzio Etruria - Toscana
\tlcventry{2005}{2012}{Responsabile della contabilità}{\href{http://www.consorzioetruria.it/}{Consorzio Etruria}}{Montelupo F.no (FI)}{}{Contabilità attiva e passiva di importanti Centri Commerciali realizzati in Toscana per UNICOOP Firenze:%
\begin{tightitemize}
 \item dal 2005 al 2007 centro comm.le CENTRO*Empoli;
 \item dal 2007 al 2009 centro comm.le PARCO*Prato;
 \item dal 2009 al 2011 magazzino logistico Centro Freschi di Pontedera;
 \item Contabilità attiva verso la committente per un totale di oltre \textbf{155} milioni di euro, contabilità passiva verso fornitori e subappaltatori per un totale di circa \textbf{128} milioni di euro, studio e redazione di varianti in corso d'opera per circa \textbf{25} milioni di euro;
 \item Addetto al controllo di gestione, con procedure per la rilevazione di S.I.L. e cost-control sviluppate personalmente;
 \item Assistente di cantiere, con funzioni di ripreventivazione del budget di commessa, preventivazione acquisti, controllo manodopera, preposto alla sicurezza.
\end{tightitemize}}
% Consorzio Etruria - Puglia
\tlcventry{2002}{2005}{Direttore tecnico}{\href{http://www.consorzioetruria.it}{Consorzio Etruria - Direzione Fuori Sede Puglia}}{Lecce}{}{Gestione globale delle Commesse:
\begin{tightitemize}
 \item Completamento delle sistemazioni esterne dell'I.P.A. di Santa Cesarea Terme, S.A. Provincia di Lecce;
 \item Costruzione di un ``centro gestionale e posto di ristoro'' nella Z.I. di Brindisi, S.A. S.I.S.R.I. Brindisi;
 \item Costruzione della ``sede decentrata degli uffici comunali'' di Gallipoli, tramite procura speciale dell'impresa Polistrade - Campi Bisenzio (FI);
 \item Costruzione dell'I.P. Alberghiero di Santa Cesarea Terme, tramite procura speciale del Consorzio Cooperativo Costruzioni - Bologna;
 \item Addetto al controllo di gestione per tutti i cantieri afferenti alla direzione fuori sede del Consorzio Etruria di Lecce per un importo totale dei lavori di oltre \textbf{40} milioni di euro.
\end{tightitemize}}
% E.R. & B. Consulting
\tlcventry{1999}{2001}{Assistente di cantiere}{E.R.\& B. Consulting}{Lecce}{}{Tecnico di cantiere, gestione maestranze e mezzi, contabilità di cantiere, preventivazione acquisti nei seguenti lavori di restauro, cat. OG2, nel centro storico di Lecce, S.A. Comune di Lecce:
\begin{tightitemize}%
 \item Ex Conservatorio di S.Anna ed Ex Monastero dei Teatini, Lecce;
 \item Recupero di Palazzo Turrisi-Palumbo, ditta appaltatrice Faesulae;
 \item Lavori di restauro dell'ex Monastero degli Olivetani, S.A. Università degli Studi di Lecce;
 \item Addetto all'ufficio gare con lo studio di circa venti gare d' appalto tra il 2000 e il 2001;
 \item Progettazione, preventivazione e successiva gestione del parco hardware e software delle rete di computer degli uffici (6 postazioni + 1 server). 
\end{tightitemize}}
% STIGEA Coop.
 \tlcventry{1998}{1999}{Disegnatore CAD}{STIGEA Cooperativa a r.l.}{L'Aquila}{}{Collaboratore esterno alla società operante nel settore dei lavori stradali per l'ampliamento della s.s. n. 7 Brindisi-Taranto - III lotto:
\begin{tightitemize}%
 \item Rilievi topografici, restituzione su CAD, produzione degli allegati contabili e computi metrici per la redazione della contabilità dei lavori eseguiti da allegare ai S.A.L.; 
\end{tightitemize}}

% Restore normal labels
%\tltext{\scriptsize}

%--------
%% Capacità e competenze
%--------

%informazione estese prese da capacita-competenze.text
%commentare per avere il curriculum sintetico

\section{Capacit\`a e competenze}

\subsection{Capacit\`a professionali}
\cvline{Diagnosi Energetica}{Audit energetico degli edifici, con acquisizione dati catastali ed esecuzione di accurato sopralluogo e rilievo. Analisi mirata dell'involucro edilizio e degli impianti. Inserimento dati e calcolo dei vari indici per determinare l'indice di prestazione energetica globale EP\ped{gl,nren} mediante software certificato C.T.I. Emissione, consegna e deposito dell'APE. Valutazione, studio ed individuazione degli interventi di riqualificazione energetica, dei risparmi economici ottenibili e dei tempi di ritorno degli investimenti. Valutazioni su progettazione/integrazione impiantistica con impianti da F.E.R. e redazione pratiche per accesso agli incentivi fiscali. Analisi e calcolo dei ponti termici e studio degli ombreggiamenti. Analisi parametriche in regime dinamico.}
\cvline{Direzione di Cantiere}{Gestione globale della commessa, con organizzazione del cantiere, gestione manodopera e mezzi, programmazione dei lavori. Ottima conoscenza dei materiali da costruzione e delle principali tecniche costruttive oltre che di restauro conservativo e consolidamento. Preventivazione acquisti, conoscenza delle normative sia dei lavori privati che pubblici, corretti rapporti con la Committenza, DD.LL. e Responsabili del Procedimento, gestione efficace di subappaltatori e fornitori.}
\cvline{Contabilit\`a di Cantiere}{Uso degli strumenti di misura, tenuta di brogliacci delle misure, rapidit\`a nelle operazioni di rilevo in cantiere, uso avanzato di fogli di calcolo e di software di contabilit\`a, tenuta dei registri contabili. Studio e redazione dei Capitolati Speciali d'Appalto, individuazione dei lavori in variante, iscrizione di riserve, analisi dei prezzi, formazione di liste per lavori in economia, studio gare di appalto.}
\cvline{Controllo di Gestione}{Formazione del budget di commessa e del quadro economico di riferimento, controllo costi ed aggiornamento costante del quadro economico, individuazione dei ratei professionisti/fornitori/subappaltatori, individuazione S.I.L., previsone emissione S.A.L., redazione del piano finanziario e del cash-flow, individuazione Margine Operativo Lordo, reportistica avanzata sull'andamento lavori.}


\subsection{Competenze organizzative e gestionali}
\cvline{}{Sono stato Direttore Tecnico di Cantiere di una decina di importanti lavori pubblici in Puglia, gestendo pi\`u cantieri in contemporanea. Ho coordinato le maestranze, i fornitori ed i subappaltatori anche in situazioni ambientali difficili. Efficaci sono statI i rapporti con le DD.LL. e la collaborazione con gli impiegati di sede. Ho piena autonomia organizzativa ed ho sempre conquistato la fiducia dei Direttori di Produzione con cui ho lavorato. Nei grandi appalti privati di cui sono stato il Responsabile della Contabilit\`a, ho avuto collaboratori giovani che ho sempre spronato a dare il massimo ed ho contribuito a far crescere professionalmente. L'organizzazione del mio ufficio si \`e sempre distinta per il controllo e l'aggiornamento dei costi ed il rispetto di tempi ed obiettivi.}

\subsection{Competenze comunicative}
\cvline{}{Buone capacit\`a di comunicare efficacemente sia con il mio gruppo di lavoro sia con le "controparti" (Enti pubblici o committenti privati, Direzione Lavori, ditte subappaltrici o fornitrici). Preferisco sempre mettere in chiaro, sia verbalmente sia mettendolo nero su bianco, problematiche, aspetti tecnico/economici e tutto quanto di rilevante per la gestione ordinata del rapporto di lavoro. Ciò mi ha sempre consentito una gestione degli aspetti tecnici e della contabilit\`a precisa ed accurata.}

\subsection{Competenze informatiche}
\cvcomputer{Sistemi Operativi}{Microsoft Windows, Apple OSX/iOS, GNU/Linux, Android}{Office}{Word, Excel, Pages, Numbers, Open/LibreOffice, GIMP, \LaTeX }
\cvcomputer{Disegno CAD/CAM e Progettazione BIM}{AutoCAD, DraftSight, Revit}{Modellazione 3D}{ArchiCAD, Trimble Sketchup, 3DS Max, Blender}
\cvcomputer{Project Management}{Microsoft Project, Planner}{DataBase}{Microsoft Access, MySQL, MongoDB}
\cvcomputer{Contabilit\`a dei lavori}{ACCA Primus, STR}{Hardware}{Gestione di reti di computer, assemblaggio e configurazione PC}
\cvcomputer{Sviluppo software}{Bash scripting language, Python, Perl, JavaScript, Nodejs}{Certificazione energetica}{CNR Docet, ProCasaClima, Aermec MC11300, Namirial Termo}
\cvcomputer{Ponti termici, Energy~Model, Retrofit}{LBL Therm, EERE EnergyPlus, NREL Openstudio}{Energie rinnovabili}{RETScreen, fogli di calcolo MyGreenBuildings}

%\devnotes{Developer}{Contributor}


\pagebreak

%--------
%% Esperienze formative
%--------

\section{Esperienze formative}

\tlcventry{2018}{2019}{Mech-Energy - Corso per Tecnico della Progettazione ed Elaborazione di Sistemi di Risparmio Energetico in Azienda}{\href{http://www.its-energiaeambiente.it/it/progetti-toscana-its-energia-ambiente/item/87-progetto-mech-energy}{ITS Energia\&Ambiente}}{organizzato da Assoservizi s.r.l. Arezzo}{}{Corso di qualifica professionale della durata complessiva di \textbf{600} ore di cui 210 di stage. Territorio toscano (focus Arezzo); Fabbrica intelligente e processi eco-sostenibili; Legislazione europea, nazionale e regionale; Sistemi energetici; Tecnologie per l’efficientamento energetico; Economia ambientale: risorse energetiche e valutazione investimenti; Efficienza e valutazione energetica; Domotica per il risparmio energetico; Sicurezza e prevenzione nei luoghi di lavoro.}

\tldatecventry{2019}{Corso Arduino}{\href{https://golem.linux.it/wiki/Corso_Arduino_2019}{Gruppo Operativo Linux EMpoli}}{Empoli (FI)}{}{Corso base di 7 lezioni sulla scheda a microcontrollore Arduino. Introduzione alla scheda Arduino; Introduzione all'elettronica; Sensori digitali ed analogici; Il PWM, motori DC e servo; Display a 7 segmenti ed LCD; I protocolli di comunicazione; Progetto finale: Termostato intelligente.}

\tlcventry{2017}{2018}{E-QUalification of the Energy Manager}{\href{http://www.e-quem.enea.it}{e-Quem}}{online}{}{Corso di formazione continua per chi lavora nel settore energia. Formazione iniziale per avvicinarsi alla professione di Tecnico esperto in gestione dell’energia. Corso articolato in 10 moduli della durata complessiva di 100 ore, erogato attraverso la piattaforma ENEA e-LEARN.}

%\tldatecventry{2017}{Financing the energy renovation of residential buildings through soft loans \& third-party investment}{\href{http://www.energy-cities.eu}{Energy Cities}}{online}{}{Webinar su alcuni case studies europei su prestiti agevolati e finanziamenti di terze parti per incentivare i cittadini al rinnovo energetico di edifici residenziali.}

\tldatecventry{2013}{Corso Base per Progettisti CasaClima}{\href{http://www.firenzenergia.it/afe/formazione.php}{Agenzia Fiorentina per l'Energia}}{Firenze}{}{Introduzione a CasaClima, fondamenti di fisica tecnica applicata e isolamento termico, materiali e costruzioni, Impiantistica.}

\tlcventry{2012}{2013}{CERT.E. - Corso per Certificatore Energetico}{\href{http://www.asev.it/formazione/archivio/certe.php}{ASEV}}{Empoli (FI)}{}{Corso di qualifica professionale della durata complessiva di \textbf{327} ore di cui 140 di stage. Normativa, fondamenti di fisica tecnica, materiali edili, costruzioni e isolamenti, impianti, metodologie di misurazione-calcolo, Project work.}

\tlcventry{2012}{2013}{Corso in Bioedilizia e Bioarchitettura}{CESVIP Toscana-Lazio}{Firenze}{}{Bioedilizia, strategie passive per il risparmio energetico in edilizia, certificazione energetica degli edifici.}

\tldatecventry{2013}{Corso per aggiornamento di formazione dei lavoratori nel settore costruzioni - Rischio alto}{ClA 1921 s.c.}{Montelupo F.no (FI)}{}{}

\tldatecventry{2013}{Corso formativo in CAD 2D - Avanzato}{CESVIP Toscana-Lazio}{Firenze}{}{}

\tldatecventry{2012}{Corso di Tecniche di conduzione e controllo di cantieri edili}{CESVIP Toscana-Lazio}{Firenze}{}{Tecniche di conduzione e controllo di cantieri edili, allestimento del cantiere, macchine e attrezzature edili, normativa sugli appalti pubblici.}

\tldatecventry{2011}{Corso su Sistemi geotermici a bassa entalpia per applicazioni a pompa di calore}{Klimahouse Umbria 2011}{}{Bastia Umbra (PG)}{}{}

%\tldatecventry{2009}{Corsi per Addetti del Servizio di Prevenzione e Protezione, prevenzione incendi e formazione teorico-pratico per addetti al primo soccorso}{Comitato Tecnico Paritetico della Toscana}{Montelupo F.no (FI)}{}{}

\tlcventry{1992}{1996}{Studente in Ingegneria Edile}{\href{http://www.poliba.it/}{Politecnico}}{Bari}{}{Frequenza di quindici corsi del Corso di Laurea, di cui undici superati gli esami.\\Votazione media 28/30.}
 
\section{Certificazioni}

\tldatecventry{2019}{Tecnico della progettazione ed elaborazione di sistemi di risparmio energetico}{\href{http://www.its-energiaeambiente.it/it/progetti-toscana-its-energia-ambiente/item/87-progetto-mech-energy}{ITS Energia\&Ambiente}}{organizzato da Assoservizi s.r.l. Arezzo}{}{ADA \#1: Attività di monitoraggio delle strutture esistenti; ADA \#2: Elaborazioni di piani di risparmio energetico; ADA \#3: Progettazione di sistemi di risparmio energetico; ADA \#4: Valutazione del piano di risparmio energetico di organizzazioni pubbliche o private.}

\tldatecventry{2014}{Esami di Stato per l'abilitazione all'esercizio della libera professione}{Istituto Statale per Geometri e Ragionieri ``Salvemini - Duca d'Aosta''}{Firenze}{}{Votazione 74/100.}

\tldatecventry{2013}{Tecnico della progettazione ed elaborazione di sistemi di risparmio energetico}{\href{http://www.asev.it/formazione/archivio/certe.php}{ASEV}}{Empoli}{}{ADA \#1: Attività di monitoraggio delle strutture esistenti.}

\tldatecventry{1992}{Diploma di Maturità Tecnica per Geometri}{ITS per Geometri ``O. Belluzzi''}{Brindisi}{}{Votazione 56/60.}

%--------
%% Lingue straniere
%--------

\section{Lingue}
\cvlanguage{Italiano}{Nativo}{Madrelingua}
\cvlanguage{Inglese}{Buon livello}{Lettura abituale di testi tecnici in inglese, pratica occasionale}
\cvlanguage{Francese}{Livello intermedio}{Studiato 3 anni a scuola, pratica occasionale}

%--------
%% Interessi personali
%--------

%\section{Interessi personali}

%\cvhobby{Sport}{Pallacanestro, praticata anche a livello agonistico. Istruttore di mini-basket.}
%\cvhobby{Altro}{Viaggi in camper, lettura, tecnologia, cucina.}

%\renewcommand{\listitemsymbol}{-} % change the symbol for lists

%--------
%% Pubblicazioni (magari un giorno...)
%--------

% Publications from a BibTeX file without multibib\renewcommand*{\bibliographyitemlabel}{\@biblabel{\arabic{enumiv}}}% for BibTeX numerical labels
%\nocite{*}
%\bibliographystyle{plain}
%\bibliography{publications}       % 'publications' is the name of a BibTeX file

% Publications from a BibTeX file using the multibib package
%\section{Publications}
%\nocitebook{book1,book2}
%\bibliographystylebook{plain}
%\bibliographybook{publications}   % 'publications' is the name of a BibTeX file
%\nocitemisc{misc1,misc2,misc3}
%\bibliographystylemisc{plain}
%\bibliographymisc{publications}   % 'publications' is the name of a BibTeX file

%--------
%% Liberatoria privacy
%--------

%liberatoria privacy aggiornata al GDPR
%de-commentare la seguente riga nel curriculum sintetico
\tiny\vspace{\fill}
\footnotesize\noindent\centering
Autorizzo il trattamento dei miei dati personali ai sensi del D.Lgs. n° 196/2003 e dell'art. 13 del Regolamento (UE) 2016/679.
%\vspace*{0.25cm}

%--------
%% Data e firma
%--------

Empoli, \today\hfill  In fede:\raisebox{-0.75cm}{\includegraphics[scale=1]{img/firma}}

%in caso di curriculum esteso metto liberatoria, data e firma non in fondo
\vfill

\end{document}

